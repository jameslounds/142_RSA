\documentclass{article}
\author{
    {\bf Jack Griffiths,}
    {\bf Max Johnson,}
    {\bf James Lounds, } \\
    {\bf Harvey Olive, }
    {\bf Oscar Oliver, }
    {\bf James Taylor}
}
\title{The RSA Algorithm}
\begin{document}
\maketitle
\newpage

\section{Introduction}
RSA is an asymmetric cryptosystem first publicly proposed by Rivest, Shamir, and Adler in 1977.
It has been used since at least 1973 in secret by intelligence organisations.
An asymmetric cryptosystem is a way to encrypt a message with a "key" that is public
- that is anyone can know its value without compromising the secuirty of the system -
and a way to decrypt the encrypted message with a \emph{private} key
- if this key is known (with the public key), it is easy for an attacker to decrypt messages.
\\
The security of the cryptosystem relies on the difficulty of factoring large prime numbers as we will see in section idk
\subsection{Original Requirements}
In their 1977 paper, Rivest, Shamir and Adler proposed the following criteria an asymmetric cryptosystem should satisfy
\\For an encyption procedure E, and decryption procedure D on a message M
\begin{enumerate}
    \item $D(E(M)) = M$
    \item $D, E$ should be easy tp compute
    \item Revealing $E$ does not reveal an efficient method for $D$
    \item $E(D(M)) = M$
\end{enumerate}
According to Diffie and Hellman's paper [New Directions in Cryptography 1976], satisfying (1)-(3) implies $E$ is a "Trap-Door One-Way Function".
With the added criterion of (4), $E$ must be  a "Trap-Door One-Way Permutation" - each output is a valid input tp the function
A One-Way function is easy to compute one way, but hard to compute the inverse.
A Trap-Door function has a hard to compute inverse, unless some private information is known, which makes the inverse easy to compute

\end{document}