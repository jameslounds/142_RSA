\documentclass{article}
\usepackage{cite}
\usepackage{amsmath, amssymb, amsthm}
\usepackage{mathtools}
\author{
    {\bf Jack Griffiths,}
    {\bf Max Johnson,}
    {\bf James Lounds, } \\
    {\bf Harvey Olive, }
    {\bf Oscar Oliver, }
    {\bf James Taylor}
}
\title{The RSA Algorithm}
\begin{document}
\maketitle
\newpage

\section{Introduction}
RSA is an asymmetric cryptosystem first publicly proposed by Rivest, Shamir, and Adler in 1977 \cite{RSA}.
It has been used since at least 1973 in secret by intelligence organisations.
An asymmetric cryptosystem is a way to encrypt a message with a "key" that is public
- that is anyone can know its value without compromising the secuirty of the system -
and a way to decrypt the encrypted message with a \emph{private} key
- if this key is known (with the public key), it is easy for an attacker to decrypt messages.
\\
The security of the cryptosystem relies on the difficulty of factoring large prime numbers as we will see in section idk
\subsection{Original Requirements}
In their 1977 paper, Rivest, Shamir and Adler proposed the following criteria an asymmetric cryptosystem should satisfy
\\For an encyption procedure E, and decryption procedure D on a message M
\begin{enumerate}
    \item $D(E(M)) = M$
    \item $D, E$ should be easy to compute
    \item Revealing $E$ does not reveal an efficient method for $D$
    \item $E(D(M)) = M$
\end{enumerate}
According to Diffie and Hellman's paper \cite{Directions}, satisfying (1)-(3) implies $E$ is a "Trap-Door One-Way Function".
With the added criterion of (4), $E$ must be  a "Trap-Door One-Way Permutation" - each output is a valid input to the function
A One-Way function is easy to compute one way, but hard to compute the inverse.
A Trap-Door function has a hard to compute inverse, unless some private information is known, which makes the inverse easy to compute

\newpage
\section{The Proposed Cryptosystem}
\subsection{Definitions}
Preamble
\[ E(M) \equiv M^e \: (mod \: n),\;\;\; D(M) \equiv M^d \: (mod \: n)\]
with $ n \coloneqq p \cdot q $ and $p, q$ are prime, and $e^{-1} \equiv d \: (mod \: \phi (n))$. \\
In order to generate a private and public key, we first need to find 2 large primes $p, q$.
It is important they remain secret as if an attacker were able to factorise
$ n \coloneqq p \cdot q $,
they would easily be able to compute the decryption key. We can then choose a random $d$ such that $GCD(d, \phi (n)) = 1$,
that is d has an inverse $(mod \: \phi(n))$.
Since $p, q$ are prime, $\phi(n) = (p-1) \cdot (q-1) $.
We can then compute $e \equiv d^{-1} \: (mod \: \phi(n))$,
and publicly send $(e, n)$ without revealing $d$.

\subsection{Proof}
\subsubsection{Lemma 1}
\textbf{Statement:} if $p, q$ are prime, then $\phi(p \cdot q) = (p-1) \cdot (q-1)$ \\\\
\textbf{Proof:} \\
Since $p, q$ are prime, $pq$ only has factors $\{1, p, q, p \cdot q\}$.
So for any $d$ which is \emph{not} coprime to $p \cdot q$, that is $gcd(d, p \cdot q) = k$,
then $d$ is of the form $d=k \cdot q$ or $d= k \cdot p$. \\
There are exactly $p-1$ values of $k$ such that $0 < k \cdot q < p \cdot q$,
and exactly $q-1$ values of $k$ such that  $0 < k \cdot p < p \cdot q$.
So the number of naturals less than $p \cdot q$ which \emph{are} coprime to $p \cdot q$ is $(p-1) + (q-1)$. Let's call this $\phi^{'}(n)$\\
$\phi(n)$ is the number of naturals less than $n$ which are \emph{not} coprime to $n$,
\begin{align*}
    \phi(n)        & = (n-1) - \phi^{'}(n)                           \\
    \phi(p\cdot q) & = (p \cdot q - 1) - \left( (p-1) + (q-1)\right) \\
                   & = p \cdot q - 1 - p + 1 - q + 1                 \\
                   & = p \cdot q - p - q + 1                         \\
                   & = (p-1) \cdot (q-1)  \;\;\;\; \blacksquare
\end{align*}
\subsubsection{Lemma 2}
\textbf{Statement:} Fermat's Little Theorem \\
\textbf{Proof:}\\
Later
\newpage
\textbf{Statement:} If $p, q$ and $n \coloneqq p \cdot q$, then for any $d \equiv e^{-1} \: (mod \: \phi(n))$
\[ E(M) \equiv M^e \: (mod \: n),\;\;\; D(M) \equiv M^d \: (mod \: n)
\]
\[ \Downarrow \]
\[  D(E(M)) \equiv M \: (mod \: n) \equiv E(D(M)) \: (mod \: n)
\]
\textbf{Proof:} \\
If a prime $p$ does not divide $M$,
$M^ {\phi(p)} \equiv M^ {p-1} \: (mod \: p) \equiv 1 \: (mod \: p)$ by Fermat's Little Theorem \\
Since $\phi(p)$ divides $\phi(n)$, $M^{\phi(p)} \equiv M^{k \phi(n)} \: (mod \: p) \equiv 1 \: (mod \: p)$
By multiplying by $M$, $M^p \equiv M \:(mod \: p)$.
\\ By the Chinese Remainder Theroem: \\
$M^{k \phi(p) + 1} \equiv M \: (mod \: p),\; M^{k \phi(q) + 1} \equiv M \: (mod \: q) \Rightarrow M^{k \phi (n)} \equiv M \: (mod \: n)$\\
Therefore if $e \cdot d = k \phi(n) + 1$, that is $e \cdot d \equiv 1 \: (mod \: \phi(n))$, or $e \equiv d^{-1} \: (mod \: \phi(n))$ \\
then $D(E(M)) = (M^{e})^{d} = M^{e \cdot d} \equiv M \: (mod \: n)$
\newpage
\bibliographystyle{plain} % We choose the "plain" reference style
\bibliography{refs} % Entries are in the refs.bib file

\end{document}