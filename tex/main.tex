\documentclass{article}
\usepackage{cite}
\usepackage{amsmath, amssymb, amsthm}
\usepackage{mathtools}
\newtheorem{theorem}{Theorem}
\author{
    {\bf Jack Griffiths,}
    {\bf Max Johnson,}
    {\bf James Lounds, } \\
    {\bf Harvey Olive, }
    {\bf Oscar Oliver, }
    {\bf James Taylor}
}
\title{The RSA Algorithm}
\begin{document}
\maketitle
\newpage

\section{Introduction}
RSA is an asymmetric cryptosystem first publicly proposed by Rivest, Shamir, and Adler in 1977 \cite{RSA}.
It has been used since at least 1973 in secret by intelligence organisations.
An asymmetric cryptosystem is a way to encrypt a message with a "key" that is public
- that is anyone can know its value without compromising the secuirty of the system -
and a way to decrypt the encrypted message with a \emph{private} key
- if this key is known (with the public key), it is easy for an attacker to decrypt messages.
\\
The security of the cryptosystem relies on the difficulty of factoring large prime numbers as we will see in section idk
\subsection{Original Requirements}
In their 1977 paper, Rivest, Shamir and Adler proposed the following criteria an asymmetric cryptosystem should satisfy
\\For an encyption procedure E, and decryption procedure D on a message M
\begin{enumerate}
    \item $D(E(M)) = M$
    \item $D, E$ should be easy to compute
    \item Revealing $E$ does not reveal an efficient method for $D$
    \item $E(D(M)) = M$
\end{enumerate}
According to Diffie and Hellman's paper \cite{Directions}, satisfying (1)-(3) implies $E$ is a "Trap-Door One-Way Function".
With the added criterion of (4), $E$ must be  a "Trap-Door One-Way Permutation" - each output is a valid input to the function.
A One-Way function is easy to compute one way, but hard to compute the inverse.
A Trap-Door function has a hard to compute inverse, unless some private information is known, which makes the inverse easy to compute

\subsection{Utility}
Assymetric Encryption is fundamental to how modern networks communicate securely, albeit using a different algorithm
% add a note here about ECRSA
Assymetry makes the ability to communicate securely with no private channel, and the ability to sign messages easy.
There are symmetric algorithms for both, but assymetry removes some trust - since the other party cannot give access to your private key, but could give access to a shared key.

The ability to send information about how to communicate securely over an insecure channel was vital in the development of secure email, and http protocols.
Before modern encryption standards, parties would swap some physical data, be it a hard drive, or a post it note.
From this shared data, which was known to be private,
both parties could establish shared secrets generated by the original physical secret,
using a publicly available algorithm.
This reliance on physical secrecy was not scalable for individuals to secure their communications.
It would be akin to sending Google a letter with your password on.
Diffie and Hellman \cite{Directions} propose a system for creating a shared secret over an insecure channel using modular exponentiation.

The ability to sign a message is more unique to assymetric cryptography.
If we had a symmetric system, in order for Bob to prove to Charlie that Alice sent him a message,
Bob must reveal Alice and Bob's shared secret to Charlie.
Since Charlie now knows the secret, she can decrypt any message Alice and Bob sent to one another with this key
However, in a public key system, anyone can know Alice's public key, so for Bob to prove to Charlie that Alice sent him a message,
he only needs to tell Charlie Alice's public key, maintaining the privacy of Alice's key pair.
In their paper \cite{RSA}, Rivest, Shamir and Adler propose a signing system
in which the Alice decrypts her plaintext message, and sends it to Bob (with the message).
Since $E(D(M)) = M$, Bob can encrypt the signature with Alice's public key, and check that the messages match.
This can be optimised by hashing (a one way map) the message before decrypting it. That is $S = D(\mathcal{H}(M))$
Since hashes are a fixed length we can require the hash to be smaller than the size of a message block,
and only need to transmit one more block instead of twice as many.
This signature algorithm is very rarely used, in favour of DSA, or ECDSA [citation needed].
\newpage
\section{The Proposed Cryptosystem}
\subsection{Definitions}
\[ E(M) \equiv M^e \: (mod \: n),\;\;\; D(M) \equiv M^d \: (mod \: n)\]
with $ n \coloneqq p \cdot q $ and $p, q$ are prime, and $e^{-1} \equiv d \: (mod \: \phi (n))$. \\
In order to generate a private and public key, we first need to find 2 large primes $p, q$.
It is important they remain secret as if an attacker were able to factorise
$ n \coloneqq p \cdot q $,
they would easily be able to compute the decryption key. We can then choose a random $d$ such that $GCD(d, \phi (n)) = 1$,
that is d has an inverse $(mod \: \phi(n))$.
Since $p, q$ are prime, $\phi(n) = (p-1) \cdot (q-1) $.
We can then compute $e \equiv d^{-1} \: (mod \: \phi(n))$,
and publicly send $(e, n)$ without revealing $d$.

\subsection{Proof}
\begin{theorem}
    If $p, q$ are prime, then $\phi(p \cdot q) = (p-1) \cdot (q-1)$
    \begin{proof}
        Since $p, q$ are prime, $pq$ only has factors $\{1, p, q, p \cdot q\}$, so for any $d$ which is not coprime to $p \cdot q$
        \begin{align*}
             & \text{Let }k \coloneqq gcd(d, p \cdot q) \;, k \neq 1 \text{ since } d \text{ is not coprime to } p \cdot q \\
             & d = k \cdot 1 \text{, or } d = k \cdot p \text{, or } d = k \cdot q
        \end{align*}
        In the first case, for $d$ to divide $p \cdot q$, $k$ must be $p$ or $q$ since $p, q$ are prime.
        Both of these are covered in the $d = k \cdot p$ or $d = k \cdot q$ cases. \\
        Let's define $\phi^{'}(n)$ to be the cototient of $n$,
        \[
            \phi^{'}(n) \coloneqq n - \phi(n)
        \]
        $\phi^{'}(n)$ is the number of naturals \emph{less than} or equal to n which are \emph{not} coprime to n.
        That is, $\phi^{'}(n)$ is the number of values of $k$ such that $0 < d \leq p \cdot q$.
        \begin{align*}
            0 & < k \cdot p \leq p \cdot q \Rightarrow 1 \leq k < q \Rightarrow k \in \left\{ 2, 3, \ldots, q \right\} \text{ since } k \neq 1 \\
            0 & < k \cdot q \leq p \cdot q \Rightarrow 1 \leq k < p \Rightarrow k \in \left\{ 2, 3, \ldots, p \right\} \text{ since } k \neq 1 \\
            0 & < k \cdot 1 \leq p \cdot q \Rightarrow k \in \left\{ p,\: q,\: p \cdot q \right\}
        \end{align*}
        So $\phi^{'}(n)$ is the sum of the cardinality of these sets.
        \begin{align*}
            \phi^{'}(p \cdot q)         & = \left| p,\: q,\: p\cdot q \right| + \left| 2, 3, \ldots, p  \right| + \left| 2, 3, \ldots, q \right| \\
                                        & = 3 + (p-1) + (q-1)  = p + q - 1                                                                       \\
            \Rightarrow \phi(p \cdot q) & = p \cdot q - (p + q - 1) = p \cdot q - p - q + 1                                                      \\
                                        & = (p-1) \cdot (q-1)
        \end{align*}

    \end{proof}
\end{theorem}

\begin{theorem}
    Fermat's Little Theorem
    \begin{proof}
        To be done
    \end{proof}
\end{theorem}

\begin{theorem}
    If $p, q$ are prime and $n \coloneqq p \cdot q$, then for any $d \equiv e^{-1} \: (mod \: \phi(n))$
    \[ E(M) \equiv M^e \: (mod \: n),\;\;\; D(M) \equiv M^d \: (mod \: n)
    \]
    \[ \Downarrow \]
    \[  D(E(M)) \equiv M \: (mod \: n) \equiv E(D(M)) \: (mod \: n)
    \]
    \begin{proof}
        If a prime $p$ does not divide $M$,
        \begin{align*}
            M^ {\phi(p)} & \equiv M^ {p-1} \: (mod \: p) \equiv 1 \: (mod \: p) \tag{\emph{by Fermat's Little Theorem}}             \\
                         & \equiv M^{k \phi(n)} \: (mod \: p) \equiv 1 \: (mod \: p) \tag{\emph{Since $\phi(p)$ divides $\phi(n)$}} \\
            M^{p}        & \equiv M ^{k \phi(n) + 1}\: (mod \: p) \equiv M \: (mod \: p) \tag{\emph{By multiplying by $M$}}
        \end{align*}
        The same can be argued for $q$ \\\\
        By the Chinese Remainder Theroem: \\
        \[
            \begin{rcases*}
                M^{k \phi(n) + 1} \equiv M \: (mod \: p) \\
                M^{k \phi(n) + 1} \equiv M \: (mod \: q)
            \end{rcases*}
            \Rightarrow
            M^{k \phi (n) + 1} \equiv M \: (mod \: n)
        \]
        By definition, $e \equiv d ^{-1} \: (mod \: n)$, so
        \begin{align*}
            e \cdot d           & \equiv 1 \: (mod \: \phi(n))                      \\
            e \cdot d           & = k \phi(n) + 1 \tag{for some $k \in \mathbb{N}$} \\
            \Rightarrow D(E(M)) & \equiv M ^ {e \cdot d} \: (mod \: n)              \\
                                & \equiv M^{k \cdot \phi(n) + 1} \: (mod \: n)      \\
                                & \equiv M \: (mod \: n)
        \end{align*}
    \end{proof}
\end{theorem}

\newpage
\bibliographystyle{plain} % We choose the "plain" reference style
\bibliography{refs} % Entries are in the refs.bib file

\end{document}